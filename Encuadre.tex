% Options for packages loaded elsewhere
\PassOptionsToPackage{unicode}{hyperref}
\PassOptionsToPackage{hyphens}{url}
%
\documentclass[
  11pt,
  ignorenonframetext,
]{beamer}
\usepackage{pgfpages}
\setbeamertemplate{caption}[numbered]
\setbeamertemplate{caption label separator}{: }
\setbeamercolor{caption name}{fg=normal text.fg}
\beamertemplatenavigationsymbolsempty
% Prevent slide breaks in the middle of a paragraph
\widowpenalties 1 10000
\raggedbottom
\setbeamertemplate{part page}{
  \centering
  \begin{beamercolorbox}[sep=16pt,center]{part title}
    \usebeamerfont{part title}\insertpart\par
  \end{beamercolorbox}
}
\setbeamertemplate{section page}{
  \centering
  \begin{beamercolorbox}[sep=12pt,center]{part title}
    \usebeamerfont{section title}\insertsection\par
  \end{beamercolorbox}
}
\setbeamertemplate{subsection page}{
  \centering
  \begin{beamercolorbox}[sep=8pt,center]{part title}
    \usebeamerfont{subsection title}\insertsubsection\par
  \end{beamercolorbox}
}
\AtBeginPart{
  \frame{\partpage}
}
\AtBeginSection{
  \ifbibliography
  \else
    \frame{\sectionpage}
  \fi
}
\AtBeginSubsection{
  \frame{\subsectionpage}
}
\usepackage{amsmath,amssymb}
\usepackage{lmodern}
\usepackage{iftex}
\ifPDFTeX
  \usepackage[T1]{fontenc}
  \usepackage[utf8]{inputenc}
  \usepackage{textcomp} % provide euro and other symbols
\else % if luatex or xetex
  \usepackage{unicode-math}
  \defaultfontfeatures{Scale=MatchLowercase}
  \defaultfontfeatures[\rmfamily]{Ligatures=TeX,Scale=1}
\fi
\usetheme[]{metropolis}
% Use upquote if available, for straight quotes in verbatim environments
\IfFileExists{upquote.sty}{\usepackage{upquote}}{}
\IfFileExists{microtype.sty}{% use microtype if available
  \usepackage[]{microtype}
  \UseMicrotypeSet[protrusion]{basicmath} % disable protrusion for tt fonts
}{}
\makeatletter
\@ifundefined{KOMAClassName}{% if non-KOMA class
  \IfFileExists{parskip.sty}{%
    \usepackage{parskip}
  }{% else
    \setlength{\parindent}{0pt}
    \setlength{\parskip}{6pt plus 2pt minus 1pt}}
}{% if KOMA class
  \KOMAoptions{parskip=half}}
\makeatother
\usepackage{xcolor}
\newif\ifbibliography
\setlength{\emergencystretch}{3em} % prevent overfull lines
\providecommand{\tightlist}{%
  \setlength{\itemsep}{0pt}\setlength{\parskip}{0pt}}
\setcounter{secnumdepth}{-\maxdimen} % remove section numbering
\ifLuaTeX
  \usepackage{selnolig}  % disable illegal ligatures
\fi
\IfFileExists{bookmark.sty}{\usepackage{bookmark}}{\usepackage{hyperref}}
\IfFileExists{xurl.sty}{\usepackage{xurl}}{} % add URL line breaks if available
\urlstyle{same} % disable monospaced font for URLs
\hypersetup{
  pdftitle={Análisis y Modelado Espacial},
  pdfauthor={Gerardo Martín},
  hidelinks,
  pdfcreator={LaTeX via pandoc}}

\title{Análisis y Modelado Espacial}
\subtitle{Encuadre}
\author{Gerardo Martín}
\date{2022-06-29}

\begin{document}
\frame{\titlepage}

\begin{frame}{Sobre mí}
\protect\hypertarget{sobre-muxed}{}
\begin{itemize}
\item
  Gerardo Martín
\item
  Veterinario
\item
  MSc Biología de Conservación

  \begin{itemize}
  \tightlist
  \item
    Modelación matemática de transmisión de enfermedades entre animales
    silvestres y domésticos
  \end{itemize}
\item
  Doctorado en Salud Pública

  \begin{itemize}
  \tightlist
  \item
    Modelación del riesgo de enfermedades emergentes de murciélagos
  \end{itemize}
\end{itemize}
\end{frame}

\begin{frame}{Contacto}
\protect\hypertarget{contacto}{}
\begin{itemize}
\tightlist
\item
  \href{mailto:gerardo.mmc@enesmerida.unam.mx}{\nolinkurl{gerardo.mmc@enesmerida.unam.mx}}
\item
  \href{https://gerardommc.github.io}{gerardommc.github.io}
\item
  \href{https://www.researchgate.net/profile/Gerardo-Martin}{ResearchGate}
\end{itemize}
\end{frame}

\begin{frame}{Sobre usted\_s}
\protect\hypertarget{sobre-usted_s}{}
\begin{itemize}
\tightlist
\item
  ¿Cómo se llaman?
\item
  ¿De dónde vienen?
\item
  ¿Qué les interesa?
\item
  ¿En qué se imaginan trabajando?
\item
  ¿Qué esperan del curso?
\end{itemize}
\end{frame}

\begin{frame}{Sobre el curso}
\protect\hypertarget{sobre-el-curso}{}
\begin{itemize}
\tightlist
\item
  Introducción al modelado espacial
\item
  Análisis de la asociación espacial entre fenómenos
\item
  Modelado de áreas de distribución y nichos ecológicos
\item
  Discusión y conclusiones
\end{itemize}
\end{frame}

\begin{frame}{Contenidos del curso}
\protect\hypertarget{contenidos-del-curso}{}
\begin{enumerate}
\tightlist
\item
  Introducción
\item
  Análisis de asociación espacial
\item
  Modelación de áreas de distribución y nichos ecológicos
\item
  Discusión y conclusiones
\end{enumerate}
\end{frame}

\begin{frame}{Seminario}
\protect\hypertarget{seminario}{}
\begin{itemize}
\tightlist
\item
  Modelación de nichos y áreas de distribución:
\end{itemize}

\textbf{Efectos del cambio ambiental en la distribución de animales de
importancia médica y los accidentes provocados por estos en México}
\end{frame}

\begin{frame}{Profesorxs participantes}
\protect\hypertarget{profesorxs-participantes}{}
\begin{itemize}
\item
  Dra. Iris Neri Flores \(\rightarrow\) Aplicaciones de modelos
  digitales de elevación
\item
  Dr.~Gustavo Martín

  \begin{itemize}
  \tightlist
  \item
    Análisis multicriterio, Calidad de datos, Cambio de uso y cobertura
  \end{itemize}
\end{itemize}
\end{frame}

\begin{frame}{Lo feo \ldots{} las evaluaciones}
\protect\hypertarget{lo-feo-las-evaluaciones}{}
\begin{itemize}
\item
  Evaluación diangóstica
\item
  Evaluación formativa

  \begin{itemize}
  \tightlist
  \item
    Trabajos y retroalimentación
  \end{itemize}
\item
  Exámenes
\item
  Asistencia
\item
  Participación
\end{itemize}
\end{frame}

\begin{frame}{Las evaluaciones}
\protect\hypertarget{las-evaluaciones}{}
\begin{itemize}
\tightlist
\item
  Trabajos 50\%
\item
  Examenes 25\%
\item
  Asistencia 25\%
\item
  Participación \(\rightarrow\) extra
\end{itemize}
\end{frame}

\begin{frame}{Estrategias pedagógicas}
\protect\hypertarget{estrategias-pedaguxf3gicas}{}
\begin{itemize}
\item
  Google Classroom: ifpezz3
\item
  Presentaciones
\item
  Prácticas en clase

  \begin{itemize}
  \tightlist
  \item
    Calculadora
  \item
    Geogebra
  \item
    Excel
  \item
    R (introducción)
  \end{itemize}
\item
  Prácticas de tarea

  \begin{itemize}
  \tightlist
  \item
    Entregadas en GC
  \end{itemize}
\end{itemize}
\end{frame}

\hypertarget{preguntas}{%
\section{Preguntas?}\label{preguntas}}

\end{document}
